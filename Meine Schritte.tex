\begin{document}
\section{Recherche}
Literatur Paper

\section{1. Datensatz breastcancer}
1. Schritt von Sponge - Regression
-> SPONGE::functionname
-> davon Parameter coeffizient auf >0 gesetzt
-> zuvor standardisierung:

library(matrixStats)

#Die Daten sind schon log 2 normalisiert, das merkst du daran dass es negative values gibt. Man kann aber nicht negativ reads zaehlen.

#Wir brauchen erst mal einen Varianzfiler, manche Gene sind nicht detektiert und haben den Minimalvalue -9,... Diese Gene koennen nicht skaliert werden weil man sonst durch 0 teilen wuerde
cancer_gene_expr <- cancer_gene_expr[,-which(colVars(cancer_gene_expr) < 1e-5)]

#Jetzt normalisieren wir die Genexpressionsvektoren noch auf die Normalverteilung so dass alle mean 0 und Standardabweichung 1 haben. Das erreicht man durch subtrahieren vom mean und teilen durch die SD. in R:

cancer_gene_expr_scaled <- scale(cancer_gene_expr)

#check ob es funktioniert hat
hist(colMeans(cancer_gene_expr)) 
hist(colMeans(cancer_gene_expr_scaled)) #standardisiert

hist(colSds(cancer_gene_expr)) 
hist(colSds(cancer_gene_expr_scaled)) #standardisiert

#auf den normalisierten / standardisierten Daten kannst du jetzt die Regressionskoeffizienten vergleichen

#das gleiche musst du natuerlich auch noch fuer die miRNA Daten machen!

29.11.18
-> doParallel auf 2 Kerne laufen lassen

05.11.18 
-> regex um nur von ENSG bis punkt alles zu speichern bei cancer_gene_expr_scaled:
    colnames(cancer_gene_expr_scaled) <- substr(colnames(cancer_gene_expr_scaled), 1, 15)
    
->   genes_miRNA_candidates <- sponge_gene_miRNA_interaction_filter(
     gene_expr = cancer_gene_expr_scaled,
     mir_expr = cancer_mir_expr_scaled,
     mir_predicted_targets = NULL,
     coefficient.threshold = 0,
     coefficient.direction = ">",
     F.test = TRUE)
Ergebnis hier: 




\end{document}
